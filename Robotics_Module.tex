

% Document settings
\documentclass{beamer}
\usepackage{listings}
\usepackage{color}
\begin{document}

\definecolor{mygreen}{rgb}{0,0.6,0}
\definecolor{mygray}{rgb}{0.5,0.5,0.5}
\definecolor{mymauve}{rgb}{0.58,0,0.82}

\lstset{ %
  backgroundcolor=\color{white},   % choose the background color; you must add \usepackage{color} or \usepackage{xcolor}
  basicstyle=\tiny,     % the size of the fonts that are used for the code
  breakatwhitespace=false,         % sets if automatic breaks should only happen at whitespace
  breaklines=true,                 % sets automatic line breaking
  captionpos=b,                    % sets the caption-position to bottom
  commentstyle=\color{mygreen},    % comment style
  deletekeywords={...},            % if you want to delete keywords from the given language
  escapeinside={\%*}{*)},          % if you want to add LaTeX within your code
  extendedchars=true,              % lets you use non-ASCII characters; for 8-bits encodings only, does not work with UTF-8
  keepspaces=true,                 % keeps spaces in text, useful for keeping indentation of code (possibly needs columns=flexible)
  language=C,                 % the language of the code
  keywordstyle=\bfseries\color{green!40!black},
  identifierstyle=\color{blue},
  stringstyle=\color{orange},
  morekeywords={*,...},            % if you want to add more keywords to the set
  numbers=left,                    % where to put the line-numbers; possible values are (none, left, right)
  numbersep=5pt,                   % how far the line-numbers are from the code
  numberstyle=\tiny\color{mygray}, % the style that is used for the line-numbers
  rulecolor=\color{black},         % if not set, the frame-color may be changed on line-breaks within not-black text (e.g. comments (green here))
  showspaces=false,                % show spaces everywhere adding particular underscores; it overrides 'showstringspaces'
  showstringspaces=false,          % underline spaces within strings only
  showtabs=false,                  % show tabs within strings adding particular underscores
  stepnumber=2,                    % the step between two line-numbers. If it's 1, each line will be numbered
  tabsize=2,                       % sets default tabsize to 2 spaces
 }



\title[Crisis] % (optional, only for long titles)
{Robotics Module}
\subtitle{From Snap to RobotC}
\includegraphics[scale=0.5]{EPGY.png}
\author[Author] % (optional, for multiple authors)
{Omoju Miller}
\institute[Stanford EPGY] % (optional)

\date[KPT 2013] % (optional)
{EPGY, Summer 2013}
\subject{Informatik}


\frame{\titlepage}

  \begin{frame}
    \frametitle{Programming Tangible Devices}
    %
    \begin{itemize}
\item In this module, we will be learning about programming Lego MindStorms NXT robots with the robotC programming language
\item So How do we go from Snap to RobotC?
\end{itemize}
  \end{frame}
  
\begin{frame}
\frametitle{C Programming}
	\begin{itemize}
		\item So there are some major differences between Snap and RobotC
		\item For one Snap is a blocks Environment, while RobotC is a text based programming language.  
	\end{itemize}
\end{frame}

\begin{frame}
\frametitle{The simplest C program }
The simplest C program is shown below. Every program must have at least this much in it. 
\lstinputlisting[language=C, firstline=1, lastline=4]{main.c}
\end{frame}

\begin{frame}
\frametitle{C Programs Contd}
Here is a much more interesting piece of code: 
\lstinputlisting[language=C, firstline=1, lastline=7]{main2.c}
When executed, the above program prints i is 11.
\end{frame}

\begin{frame}
\frametitle{Simple Program Analysis}
Let's take a minute and analyze the program on the previous slide. 
\begin{itemize}
\item Line 1 is a preprocessor directive. In this case, it tells the compiler to add some extra code that is stored in a different file when the program is compiled. 
\item Line 2 is where the program starts. The word main represents the beginning of a function. 
\item Line 3 is a definition of a variable named i. 
\item Line 4 assigns the variable i to the value 11.
\item Line 5 is responsible for printing i is 11. 
\end{itemize}

\end{frame}

\begin{frame}
\frametitle{Types of Programming Languages}
Many different languages can be used to program a computer. 
\begin{itemize}
\item Machine Language.
	\begin{itemize}
	\item A collection of detailed, cryptic instructions that control  the computer�s internal circuitry.
	\item Strings of numbers giving machine specific instructions
	\item Example:
		\begin{itemize}
			\item +1300042774
			\item +1400593419
			\item +1200274027
		\end{itemize}
	\end{itemize}
\item Assembly languages
	\begin{itemize}
	\item English-like abbreviations representing elementary computer operations (translated via assemblers)
	\item Example:
		\begin{itemize}
		\item LOAD   BASEPAY
		\item ADD    OVERPAY
		\item STORE  GROSSPAY
		\end{itemize}
	\end{itemize}
\end{itemize}
\end{frame}

\begin{frame}
\frametitle{Types of Programming Languages Contd}
\begin{itemize}
\item High-Level language
	\begin{itemize}
	\item A collection of instructions that is more compatible with human languages and human thought processes.
	\item Languages such as C, Pascal, BASIC, C++, Java and so on
	\item Example:
		\begin{itemize}
		\item grossPay = basePay + overTimePay
		\end{itemize}
	\end{itemize}
\end{itemize}
\end{frame}

\begin{frame}
\frametitle{Snap $\rightarrow$ RobotC}
\includegraphics[scale=0.5]{Robot_wait.png}
\lstinputlisting[language=C, firstline=32, lastline=37]{nxt_move_forward.c}
\end{frame}

\begin{frame}
\frametitle{What does this code do?}
\lstinputlisting[language=C, firstline=1, lastline=14]{nxt_mod.c}
\end{frame}

\begin{frame}
\frametitle{Here is how we do loops}
\includegraphics[scale=0.5]{Robot_loop_pic.png}
\lstinputlisting[language=C, firstline=1, lastline=12]{robot_loop.c}
\end{frame}

\begin{frame}
\frametitle{Download your first program}
\begin{itemize}
\item Google robotC setup download sample program NXT, to see the instructions to download your sample program
\item RobotC Sample Programs (NXT)
	\begin{itemize}
	\item This gives you links to all the sample programs provided with the package.
	\end{itemize}
\end{itemize}
\end{frame}

\begin{frame}
\frametitle{Moving Forward Code -\textit{ Dissection}}
\includegraphics[scale=0.4]{Robotics_1.png}
\end{frame}

\begin{frame}
\frametitle{Moving Forward Code -\textit{ Dissection 2}}
\includegraphics[scale=0.4]{move_forward_2.png}
\end{frame}


\begin{frame}
\frametitle{Sense Plan Act (SPA)}

Sense, Plan, Act was an early robot control procedure commonly abbreviated \textbf{SPA}. Today we 
use its fundamental concepts to remind us of the \textbf{three critical capabilities} that every robot 
must have in order to operate effectively

\begin{description}
\item[ SENSE:] The robot needs the ability to \textbf{sense important things about its environment}, like 
the presence of obstacles or navigation aids. What information does your robot need 
about its surroundings, and how will it gather that information?
\item[PLAN:] The robot needs to take the sensed data and figure out how to \textbf{respond appropriately}
to it, based on a \textbf{pre-existing strategy}. Do you have a strategy? Does your program 
determine the appropriate response, based on that strategy and the sensed data?
\item[ACT:] Finally, the robot must actually act to \textbf{carry out the actions} that the plan calls for. 
Have you built your robot so that it can do what it needs to, physically? Does it 
actually do it when told?
\end{description}
\end{frame}

\begin{frame}
\includegraphics[scale=0.5]{SPA.png}
\end{frame}


\begin{frame}
\frametitle{Boolean Logic }
\includegraphics[scale=0.4]{Bool1.png}
\end{frame}

\begin{frame}
\frametitle{Boolean Logic }
\includegraphics[scale=0.4]{Bool2.png}
\end{frame}

\begin{frame}
\frametitle{Boolean Logic }
\includegraphics[scale=0.4]{Bool3.png}
\end{frame}

\end{document}




