

% Document settings
\documentclass[11pt]{article}
\usepackage[margin=1in]{geometry}
\usepackage[pdftex]{graphicx}
\usepackage{multirow}
\usepackage{setspace}
\pagestyle{plain}
\setlength\parindent{0pt}

\begin{document}

% Course information
\includegraphics[scale=0.5]{EPGY.png}\\
{\LARGE \textsc{Introduction to Computational Thinking using Snap}!}\\
Instructor: Omoju Miller \\ omojumiller@gmail.com \\ Summer 2013.
\vspace{15mm}


\begin{flushright}
{\footnotesize CS + X for all X \\
Peter Norvig \\}
\line(1,0){250}
\end{flushright}

\vspace{3mm}

% Course details
\textbf {\large \\ Course Description:} Students will be introduced to the fundamental thinking processes that underlie Computer Science through the use of a blocks programming interface called Snap!. In addition, we will discuss the Big Ideas of computing and a broader understanding of the role of computing in the 21st century. This course emphasizes the interdisciplinary nature of Computer Science and helps students prepare to be active participants in the broader computing field.\\ This course is broken up into two modules.\\
Module 1: We will be using the extended implementation of MIT's Scratch from University of California at Berkeley called Snap! (http://snap.berkeley.edu) to learn about computation thinking.\\
Module 2: We will be using robotics to enhance our understanding of the power of computing.

\newpage

% Course Outline
\textbf {\large Tentative Course Outline}:

The weekly coverage might change as it depends on the progress of the class.  However, you must keep up with the reading assignments.

\begin{table}[h!]
\normalsize % The size of the table text can be changed depending on content. Remove if desired.
\begin{tabular}{ | c | c | }
\hline
\textbf{Day} & \textbf{Content} \\
\hline
Day 1 & \begin{minipage}{.85\textwidth}
\begin{itemize} \itemsep-0.4em
	\vspace{1mm}
	\item Intro to Blocks Programming and Big Ideas in Computer Science
\item Lab assignment: Student should create a GitHub Account
\item Lab assignment: Students should watch the BYOB tutorial to get acquainted with the environment at \\ http://www.youtube.com/watch?v=Aub6BAxAT-c\&feature=share\&list=PLAE5AE3CD22628741
\item Lab assignment: Students should play around with the Piano Project at the Scratch Website: scratch.mit.edu/projects/10012676/\#editor
	\vspace{1mm}
\end{itemize}
\end{minipage} \\
\hline
Day 2 & \begin{minipage}{.85\textwidth}
\begin{itemize} \itemsep-0.4em
	\vspace{1mm}
	\item Random, If and Input
	\item Introduction to Snap! aka BYOB
	\item Project Introduction - Fractals
	\item Lab assignment: Students should spend the time and decide on their proposal, create a presentation which they will pitch to the classmates next day
	\item Lab assignment: Students should implement a version of a music app or extend someone else's version
	\vspace{1mm}
\end{itemize}
\end{minipage} \\
\hline
Day 3 & \begin{minipage}{.85\textwidth}
\begin{itemize} \itemsep-0.4em
	\vspace{1mm}
	\item Algorithms: Solidify understanding of what an algorithm is and why they�re important to computer science and other fields.
	\item Recursion: Cursory introduction
	\item A bit about choosing and pitching a project
	\item Lab assignment: Design an algorithm to solve project Euler problem 1, Multiples of 3 and 5
	\item Lab assignment:Type the pseudocode of you algorithm and push it to your GitHub account
	\vspace{1mm}
\end{itemize}
\end{minipage} \\
\hline
Day 4 & \begin{minipage}{.85\textwidth}
\begin{itemize} \itemsep-0.4em
	\vspace{1mm}
	\item Project Presentation
	\item Recursion in Nature
	\item The simplicity of complexity
	\item Assembling Robots and helping TA's out 
	\vspace{1mm}
\end{itemize}
\end{minipage} \\
\hline
Day 5 & \begin{minipage}{.85\textwidth}
\begin{itemize} \itemsep-0.4em
	\vspace{1mm}
	\item Intro to Robotics
	\item Robot Battle team formation and project
	\item Lab Assignment: Make your robot touch the wall
	\item The robot instructions for that can be found in file \lq\lq Robot\_touch\_light\_build\_instruction.pdf\rq\rq
	\vspace{1mm}
\end{itemize}
\end{minipage} \\
\hline
Day 6 & \begin{minipage}{.85\textwidth}
\begin{itemize} \itemsep-0.4em
	\vspace{1mm}
	\item Something interesting
	\item Lab assignment: TBD
	\vspace{1mm}
\end{itemize}
\end{minipage} \\
\hline
Day 7 & \begin{minipage}{.85\textwidth}
\begin{itemize} \itemsep-0.4em
	\vspace{1mm}
	\item Something interesting
	\item Lab assignment: TBD
		\vspace{1mm}
\end{itemize}
\end{minipage} \\
\hline
Day 8 & \begin{minipage}{.85\textwidth}
\begin{itemize} \itemsep-0.4em
	\vspace{1mm}
	\item Something interesting
	\item Lab assignment: TBD
		\vspace{1mm}
\end{itemize}
\end{minipage} \\
\hline
Day 9 & \begin{minipage}{.85\textwidth}
\begin{itemize} \itemsep-0.4em
	\vspace{1mm}
	\item Robotics Project Presentation
	\item Lab Assignment: Presentation on what I have learnt about computing
	\vspace{1mm}
\end{itemize}
\end{minipage} \\
\hline
\end{tabular} 
\end{table}



\end{document}



